%%%%%%%%%%%%%%%%%%%%%%%%%%%%%%%%%%%%%%%%%
% Journal Article
% LaTeX Template
% Version 1.3 (9/9/13)
%
% This template has been downloaded from:
% http://www.LaTeXTemplates.com
%
% Original author:
% Frits Wenneker (http://www.howtotex.com)
%
% License:
% CC BY-NC-SA 3.0 (http://creativecommons.org/licenses/by-nc-sa/3.0/)
%
%%%%%%%%%%%%%%%%%%%%%%%%%%%%%%%%%%%%%%%%%

%----------------------------------------------------------------------------------------
%	PACKAGES AND OTHER DOCUMENT CONFIGURATIONS
%----------------------------------------------------------------------------------------

\documentclass[twoside]{article}

\usepackage{lipsum} % Package to generate dummy text throughout this template

\usepackage[sc]{mathpazo} % Use the Palatino font
\usepackage[T1]{fontenc} % Use 8-bit encoding that has 256 glyphs
\usepackage[utf8]{inputenc}
\linespread{1.05} % Line spacing - Palatino needs more space between lines
\usepackage{microtype} % Slightly tweak font spacing for aesthetics

\usepackage[hmarginratio=1:1,top=32mm,columnsep=20pt]{geometry} % Document margins
\usepackage{multicol} % Used for the two-column layout of the document
\usepackage[hang, small,labelfont=bf,up,textfont=it,up]{caption} % Custom captions under/above floats in tables or figures
\usepackage{booktabs} % Horizontal rules in tables
\usepackage{float} % Required for tables and figures in the multi-column environment - they need to be placed in specific locations with the [H] (e.g. \begin{table}[H])
\usepackage{hyperref} % For hyperlinks in the PDF

\usepackage{lettrine} % The lettrine is the first enlarged letter at the beginning of the text
\usepackage{paralist} % Used for the compactitem environment which makes bullet points with less space between them

\usepackage{abstract} % Allows abstract customization
\renewcommand{\abstractnamefont}{\normalfont\bfseries} % Set the "Abstract" text to bold
\renewcommand{\abstracttextfont}{\normalfont\small\itshape} % Set the abstract itself to small italic text

\newcommand{\rparen}{)}

\usepackage{titlesec} % Allows customization of titles
\renewcommand\thesection{\Roman{section}} % Roman numerals for the sections
\renewcommand{\thesubsection}{\thesection\hspace{1mm}\alph{subsection}}
\titleformat{\section}[block]{\large\scshape\centering}{\thesection}{1em}{} % Change the look of the section titles
\titleformat{\subsection}[block]{\large}{\thesubsection\rparen}{1em}{} % Change the look of the section titles

\usepackage{fancyhdr} % Headers and footers
\pagestyle{fancy} % All pages have headers and footers
\fancyhead{} % Blank out the default header
\fancyfoot{} % Blank out the default footer
\fancyhead[C]{TDT4205 Compilers $\bullet$ Assignment One $\bullet$ \date{\today}} % Custom header text
\fancyfoot[RO,LE]{\thepage} % Custom footer text

%----------------------------------------------------------------------------------------
%	TITLE SECTION
%----------------------------------------------------------------------------------------

\title{\vspace{-15mm}\fontsize{24pt}{10pt}\selectfont\textbf{Theory for Assignment One}} % Article title

\author{
\large
\textsc{Øyvind Robertsen} \\ % Your name
\normalsize Norwegian University of Science \& Technology \\ % Your institution
\normalsize \href{mailto:oyvinrob@stud.ntnu.no}{oyvinrob@stud.ntnu.no} % Your email address
\vspace{-5mm}
}
\date{}

%----------------------------------------------------------------------------------------

\begin{document}

\maketitle % Insert title

\thispagestyle{fancy} % All pages have headers and footers

%----------------------------------------------------------------------------------------
%	ABSTRACT
%----------------------------------------------------------------------------------------

%\begin{abstract}

%\noindent \lipsum[1] % Dummy abstract text

%\end{abstract}

%----------------------------------------------------------------------------------------
%	ARTICLE CONTENTS
%----------------------------------------------------------------------------------------

\begin{multicols}{2} % Two-column layout throughout the main article text

\section{Problem 1}

\lettrine[nindent=0em,lines=3]{T} he versions of gcc, flex and bison installed on \texttt{caracal.stud.ntnu.no} are listed in table 1.

\begin{table}[H]
\caption{Versions}
\centering
\begin{tabular}{ll}
\toprule
Name & Version \\
\midrule
\texttt{gcc} & $4.8.2$ \\
\texttt{flex} & $2.5.35$ \\
\texttt{bison} & $3.0.2$ \\
\bottomrule
\end{tabular}
\end{table}


%------------------------------------------------

\section{Problem 2}

\subsection{What is the difference between an interpreter and a compiler?}

Both are used to create executable code (executable either by the host machine or some virtual machine running on the host machine) from a source program.
A compiler will scan the entire program it is given and produce an executable representing the program in it's entirety.
An interpreter however, will execute code as it scans through the source program and terminate execution if an error occurs.

\subsection{What is the difference between a compiler and a translator?}

A translator performs the general act of translating between two languages.
Compilers (and interpreters) are translators used by computers to translate source code from a higher level language to a lower level one. 
Higher level languages are more easily understood by humans, while lower level ones are understood by computers.
%------------------------------------------------


%------------------------------------------------

\section{Problem 3}

\subsection*{Mention and briefly describe the different stages of a compiler.}

The stages (big picture perspective) are as follows:

\begin{compactitem}
\item Lexical Analysis
\item Syntax Analysis
\item Semantic Analysis
\item Optimization
\item Code Generation
\end{compactitem}


Lexical analysis is the process of converting a sequence of characters into a sequence of tokens.
The part of the compiler that performs this function is called a lexical analyzer, lexer or scanner.

Syntax analysis is the process of analysing a sequence of tokens in terms of some well-defined grammar describing the grammar.
The process, often called parsing, attempts to parse the tokens into some kind of in-memory representation of the input.
An often used structure is the `abstract syntax tree' (AST). Syntactically incorrect programs will be rejected at this stage.

Semantic analysis is the process of analysing the AST with regards to semantics.
This phase performs various semantic checks, like type checking, object binding, definite assignment, rejecting semantically incorrect programs and issuing warnings. During this phase, the AST is decorated with semantic information, aiding the next step.

Compiler optimization refers to a compilers attempting to minimize or maximise some attribute of the executable produced by the compiler. There are three common optimization targets; minimal execution time, minimal memory usage and maximal energy efficiency.

Code generation is the final phase of the compiler. Here, the result of the optimization phase is used to generate machine code for the final executable.
%------------------------------------------------


\end{multicols}

\end{document}
